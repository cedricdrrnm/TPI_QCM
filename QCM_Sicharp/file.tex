\documentclass[answers]{exam}
\usepackage[utf8]{inputenc}
\usepackage[demo]{graphicx}
\usepackage{amssymb}
\checkboxchar{$\square$}
\parindent=0pt

\title{test}
\author{cedric.drrnm }
\date{\today}

\begin{document}
\maketitle

\begin{minipage}{\linewidth}

\part{}
Le nombre qui suit le nombre 9 en système hexadécimal est~:
\bigskip

\begin{checkboxes}

\choice 0

\choice impossible, hexa signifie que ce système ne va pas au-delà de 6 chiffres

\choice 10

\CorrectChoice A

\end{checkboxes}

\end{minipage}

\bigskip


\begin{minipage}{\linewidth}

\part{}

La conversion binaire (sur 4 bits) du nombre hexadécimal F est~:

\bigskip


\begin{checkboxes}

\choice 1010

\choice impossible, F est trop grand pour être codé sur 4 bits

\CorrectChoice 1111

\choice 0101

\end{checkboxes}

\end{minipage}

\end{document}

%http://tex.stackexchange.com/questions/8857/how-to-type-special-accented-letters-in-latex
%http://www.grappa.univ-lille3.fr/FAQ-LaTeX/29.57.html
